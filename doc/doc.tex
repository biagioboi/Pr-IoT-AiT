% !TEX TS-program = pdflatex
% !TEX encoding = UTF-8 Unicode

\documentclass[sigconf]{acmart}
\captionsetup{font=footnotesize}
\usepackage{graphicx}

\settopmatter{printacmref=false} % Removes citation information below abstract
\renewcommand\footnotetextcopyrightpermission[1]{} % removes footnote with conference information in first column
%%\pagestyle{plain} % removes running headers
\thispagestyle{empty}
%%
%% \BibTeX command to typeset BibTeX logo in the docs
\AtBeginDocument{%
    \providecommand\BibTeX{{%
        \normalfont B\kern-0.5em{\scshape i\kern-0.25em b}\kern-0.8em\TeX}}}

\setcopyright{none}
\copyrightyear{}
\acmYear{}
\acmDOI{}

\acmConference[Computer Science]{}{University of Salerno}{UNISA}
\acmBooktitle{Privacy concerns in IoT systems with the use of AI techniques}
\acmPrice{}
\acmISBN{}

\begin{document}

    \title{[Status Report] Privacy concerns in IoT systems with the use of AI techniques}


    \author{Fabio Palomba}
    \email{fpalomba@unisa.it}
    \affiliation{
        \institution{Universit\'a degli studi di Salerno}
        \streetaddress{}
        \city{Salerno}
        \state{}
        \country{Italy}
        \postcode{}
    }

    \author{Giammaria Giordano}
    \email{ggiordano@unisa.it}
    \affiliation{%
        \institution{Universit\'a degli studi di Salerno}
        \streetaddress{}
        \city{Salerno}
        \state{}
        \country{Italy}
        \postcode{}
    }

    \author{Biagio Boi}
    \email{b.boi@studenti.unisa.it}
    \author{Gigi Jr Del Monaco}
    \email{g.delmonaco1@studenti.unisa.it}
    \affiliation{
        \institution{Universit\'a degli studi di Salerno}
        \streetaddress{}
        \city{Salerno}
        \state{}
        \country{Italy}
        \postcode{}
    }

   

    \begin{teaserfigure}
        \rule{\linewidth}{1mm}
%%  \includegraphics[width=\textwidth]{sampleteaser}
%%  \caption{Insert text here}
%%  \Description{insert description here}
%%  \label{fig:teaser}
    \end{teaserfigure}

%%
%% This command processes the author and affiliation and title
%% information and builds the first part of the formatted document.
    \maketitle


    \section{Introduction}
    The evolution of smart devices over last years has increased exponentially and the introduction of these devices in the house is progressively growing.
    The major problem related to these devices is that usually the privacy is not considered, although there are a lot of regulations (just see the GDPR) that describe how the user data have to be stored and who can access to these data.
    Starting from these two points We've decided to understand what happens within the context of smart assistance. The final idea is to develop a tool able to preserve the privacy of the users.

\section{Objective}
The goal of the project is to develop a tool based on a ML model able to preserve privacy during the information exchange in the IoT context, by analyzing the packets sent over the network.

\section{Methodology}
In order to implement the tool, we've decided to follow each of these steps:
\begin{enumerate}
\item Consider the current state of art in order to retrieve useful informations. This step is important to produce knowledge to better perform the next steps;
\item Analyze the existing datasets to achieve feature engineering;
\item Apply normalization techniques (data cleaning, data balancing);
\item Implementation and training of a ML model by considering different approaches to find the best fit model for our problem. Looking to the context related projects is most likely that we're going to focus on a Neural Network by using Keras;
\item Analyze, monitor and compare the results of each model by using ML Flow tool;
\item Develop a real time tool based on this model.
\end{enumerate}
Clearly, all these steps will be conducted using a MLOps approach.

\section{Performed Activities}
We've started to consider all the related works for the domain of the problem; we're currently analyzing the existing dataset.
\section{Planned Activities - Milestones}
\begin{itemize}
\item May 24 - Analyze the existing dataset;
\item May 28 - Feature engineering process;
\item May 31 - Data cleaning and balancing;
\item June 5 - Implementation, training and analysis of each approach;
\item June 9 - Develop a tool based on the best fit model.
\end{itemize}

    \bibliographystyle{plain}
    \bibliography{biblio_ref}

\end{document}
\endinput
%%
%% End of file `sample-sigconf.tex'.
